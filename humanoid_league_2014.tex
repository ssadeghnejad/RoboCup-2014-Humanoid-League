\documentclass{llncs}
%
\usepackage{makeidx}  % allows for indexgeneration
%
\begin{document}

\mainmatter

\title{Rule Developments in the Humanoid League from 2002 to 2014 and Perspectives for the Future}

\titlerunning{Humanoid League Perspective}

\author{Jacky Baltes\inst{1}}

\institute{University of Manitoba, Winnipeg, MB, R3T 2N2, Canada,\\
email{\ jacky@cs.umanitoba.ca},\\
WWW home page: \texttt{http://www.cs.umanitoba.ca/\~{ }jacky}
}

\maketitle

\begin{abstract}
This paper describes the major achievements in the history of the
RoboCup humanoid league from its start in 2002 to 2014. It provides a
perspective for the future of the league with a strong push towards
larger robots and human like playing fields.

The paper also describes some risks associated with these intended
changes.
\end{abstract}


\section{Introduction}

RoboCup is an international robotics competition for
robots with the stated goal of beating the human world champion in
soccer in 2050. As such, many of the competitions focus on soccer as a
challenge problem for artificial intelligence and robotics. However,
RoboCup also includes competitions for home helper, work helper, and
rescue robots.

This paper describes the history and future perspective of the RoboCup
Humanoid League. The history of the RoboCup humanoid league can be
broken up into several periods. The early years where simple walking
and kicking were formidable challenges are introduced in
section~\ref{sec:one}. As described in Sec.~\ref{sec:two}, rapid
improvements in mechanics, electronics, and algorithms resulted in
much more capable human like robots that were able to walk quickly and
robustly over the field by the end of 2007. The third period of
humanoid robot development shown in Sec.~\ref{sec:three} resulted in
smarter playing robots where winning and loosing was more determined
by a robot's capability to localize itself and map its environment
than by its walking speed. Section~\ref{sec:threeb} describes how
commercially available platforms provided teams with an opportunity to
speed up their development and the impact of those commercial
platforms on the competition.  Section~\ref{sec:four} gives an
introduction to the major changes for the RoboCup 2014 competition in
Joe Passeo, Brazil. The future evolution of the RoboCup Humanoid
League is characterized by a strong push towards larger robots, bigger
teams, and more human like soccer rules and environments as shown in
Sec.~\ref{sec:five}. Some concerns and issues with the current road
map are discussed in Sec.~\ref{sec:seven}. The paper draws conclusions
in Sec.~\ref{sec:six}.

\section{The Early Years (2002 - 2004)}
\label{sec:one}

The humanoid league is one of the youngest soccer playing leagues in
the RoboCup competition. It's inaugural event took place at RoboCup
2002 in Fukuoka, Japan.

Building a fully autonomous humanoid robot that is able to play soccer
games is still a challenging problem today, but it was clearly a blue
sky -- extremely ambitious with a high chance of failure -- project in
2002. 

At that time, there were some commercially developed humanoid robots
during that period (Honda Asimo and Sony Qrio). However, the costs of
these robots was way beyond the funding levels of the average RoboCup
participants. I remember that in 2003, a representative from Sony
introduced the Sony Qrio - the successor to the widely successful Sony
Aibo robot. In his speech, he was asked about the cost of the Qrio and
stated that the cost would be about that of a car. Many researches
were extremely excited, since the cost seemed very reasonable for such
an advanced platform. When many members of the audience said that they
would like to order one immediately, the Sony representative corrected
himself by saying: "No. You don't understand. I mean a Ferrari".

In spite of the very ambitious goal, a hodge-podge of about a dozen
teams entered the inaugural RoboCup humanoid league competition in
2002 in Fukuoka, Japan. The robot designs varied from 20cm to 180cm
tall robots. There were also many other differences between the robot
designs. Another platform that was commercially available during that
time was the Fuji HOAP series of robots. Their cost was
about \$150,000 USD, but it was not able to act fully autonomously
because it did not have sufficient processing power available
on-board. So all vision processing etc. had to be done off-board on a
PC. Another difference was that several other teams were unable to
move autonomously under battery power and had to be powered
externally. Some teams even were completely unable to act autonomously
and used remote control to move the robot towards the ball.

Due to these constraints, the first RoboCup Humanoid League consisted
of three events: balancing on one leg, free style demonstration (a
panel of judges scores a short free style demonstration by the team),
and penalty kicks. To allow these robots with very different
capabilities to compete in the same event, the RoboCup Humanoid League
TC introduced performance factors to try and level the playing
field. For example, the performance factor for remote controlled
operation was 100\%, so a goal scored with remote controlled robot
counted 50\% of a goal scored by a fully autonomous robot.

Apart from team Joitech that used the Fujitsu HOAP, all competitors
developed their own hardware for the RoboCup Humanoid League and
similar competitions (e.g., Japan Robo-One fighting robots). Since the
RoboCup Humanoid League TC realized that building your own robot was a
significant challenge at that time and since it wanted to encourage
teams to explore design ideas and build their own robots, commercial
platforms were also penalized by a 20\% performance factor.

The RoboCup Humanoid League TC was acutely aware of the fact that
building larger humanoid robots was even harder than building smaller
humanoid robots, but that to achieve the goal of 2050 large humanoid
robots are of strategical importance. Therefore, the RoboCup Humanoid
League TC separated the league into three subleagues: small (<60cm),
medium (60cm to 80cm) and large (>80cm) robots.

The constraints of these early years still influences the culture of
the humanoid league and influenced rule development in successive
years.

\section{Human Like Robots (2004 - 2008)}
\label{sec:two}

The performance of the humanoid robots in the humanoid league
developed quickly. By 2004, all robots acted fully autonomously and
all processing was handled on-board. Therefore, the need for the
performance factors vanished and they were removed from the rule
book. The rules evolved to provide fair and entertaining competitions
that could still act as benchmark problems for our research into
developing capable fully autonomous soccer robots.

As teams improved the robustness and walking ability of their robots,
it became possible to start 2 vs 2 soccer matches. The first humanoid
robot soccer matches took place as demonstrations events during
RoboCup 2003 in Padua, Italy, and RoboCup 2004 in Lisbon, Portugal, were
introduced as regular events in 2005.

The RoboCup Humanoid League TC also realized that it was important to
encourage teams to work on fundamental problems even if they were not
directly applicable to the current game play. An example are
throw-ins. In the soccer matches, throw-ins are replaced by the
referee putting the ball back into play in a neutral position without
stopping the match, since a throw-in is an often occurring event that
is a time consuming task for a humanoid robot. So, the RoboCup
Humanoid League TC decided for the faster solution (placement by the
referee) which results in more entertaining and dynamic
games. However, throw-ins will be important on our way towards 2050
and therefore the RoboCup Humanoid League TC introduced the throw-in
technical challenge so that teams can show their progress and be
rewarded by points in the technical challenge. The points in the
technical challenge do not affect the result for the soccer
competition, but are counted towards the Best Humanoid award. The
technical challenges change regularly. Other technical challenges were
uneven terrain walking, doubles passes, high kicking, and dribbling
challenges.

Many teams greatly improved the performance of their robots, but Team
VStone from Osaka, Japan, set a new bar for all competitors with their
platform. The robots were able to move quickly and stably across the
playing field. Furthermore, the robots used a small omni-vision system
in the head of the robot, which allowed the robots a 360\o{ } degree
view of the playing field. The VStone robots were extremely successful
in the soccer competitions and won the soccer competitions three times
in succession.

During this time, many formal and informal discussions were held among
the technical committee and the participants. After several years, it
became apparent that most participants felt that humanoid robots
should be limited to human like kinematics and human like sensors. As
a result of these discussions, the use of omni-vision and extreme wide
angle lenses was disallowed. Furthermore, active sensors (e.g., LADAR,
Kinect, RGBD) were also forbidden.

Due to some controversial judging results in the past, another
decision based on discussions with the teams was to remove subjective
judgment from the competition as much as possible. Quantitative
measures (e.g., goal scored) were seen as much more desirable by the
teams. Therefore, the free demonstration event was removed from the
competition.

Another effect of these changes meant that vision became the most
dominant sensor for perceiving the environment. As computer vision is
crucially dependent on the lighting conditions of the field, the rules
evolved towards more realistic vision. In 2010 extra lighting on the
field was removed, which resulted in much larger changes in brightness
due to spot lights and shadows.

During the same time frame, the performance of the larger robots also
improved significantly. Partially driven by the availability of
affordable high powered servos, the performance of the smaller adult
sized robots (80cm to 100cm and less than 10kg) had improved to the
point where 2 vs 2 soccer matches became possible. There was a strong
push from those team to introduce soccer into the adult sized
league. But, some of the larger (100cm to 130cm) and heavier robots
were still too fragile and unstable. Furthermore, since some of the
robots weighted more than 40kg, they posed a real danger to other
robots or participants should they fall.

It became apparent that there was a big difference in the technical
challenge for light 90cm tall and heavy 120cm tall humanoid
robots. Therefore, the Humanoid league TC and team leaders decided to
split the Humanoid league competitions into three size classes: kid
size robots less than 60cm tall, teen sized robots between 80cm and
120cm tall, and adult sized robots greater than 120cm.
 
\section{From Walking to Navigation (2008 - 2012)}
\label{sec:three}

In the 2008, the number of players in the soccer matches was increased
from 2 to 3 players per team. Furthermore, most teams had successfully
solved the problem of locomotion and were able to walk stably over
flat even surfaces such as hardwood floors or carpets. For these two
reasons, the localization (where is the robot?) and the mapping (where
are the ball and the other players?) became more and more
important. Whereas in previous years, a robust fast walking robot
design was sufficient to ensure success, smarter game play became
increasingly important. Two teams from Germany (Team Nimbro,
University of Freiburg and Darmstadt Dribblers, University of
Darmstadt) proved to be powerhouses during that period and won the
competition several times.

The next wave of major rule changes aimed at making localization and
mapping more realistic (the mid and corner markers were removed, the
size of the playing field extended, and detection of the goals was
made more difficult. The goals changed first by removing the uni
coloured back wall leaving only the goal posts as landmarks and by
colouring both goals in yellow (2013).

For the new adult sized league, the penalty kick was replaced by the
dribble and kick competition. Dribble and Kick is played between two
robots - a striker and the goal keeper. The striker robot starts in
the centre of the field and the ball is placed randomly on the
striker's goal box. The task of the striker is to move back to
approach the ball, dribble the ball across the centre line and then
kick the ball into the opposing goal.

\section{The DARwIn and Nimbro-OP Platforms (2012 - now)}
\label{sec:threeb}

In 2011, the Korean company Robotis introduced the DARwIn-OP robot,
which they had developed in conjunction with Prof. Dennis Hong from
Virginia Tech. A year later, a similar collaboration between Robotis
and Sven Behnke from Bonn University resulted in the development of
Nimbro-OP, a teen sized humanoid robot. The introduction of this
platforms had a big impact on the Humanoid League. Instead of
designing and building their robots from scratch, teams could now
simply purchase a robot platform, that was able to walk and kick a
ball out of the box. So it made entry into the league much easier for
new teams.

Since the number of participating teams had increased drastically over
the years, the number of participating teams in the kid size had to be
limited to 24 teams and qualification for the RoboCup competition had
become very competitive. For qualification, teams had to submit a team
description paper (TDP) and a video of their robot playing soccer. In
that video, the robot needed to demonstrate the ability to perceive
and approach a ball, line up with the goal, and to kick the ball into
the goal. It also needed to demonstrate the ability to stand up after
a fall from various positions (i.e., falling forward and falling
backwards). The DARwIn-OP had a large impact on the kid size
league. In 2014, more than 33\% of the teams that tried to qualify
used the DARwIn platform.

Many teams that built their robots from scratch felt that it was
unfair that in spite of the fact that they had spent much hard work,
time, and money on building their own robots, other teams could just
purchase a robot and qualify for the RoboCup competition with much
less effort. Other teams felt that only the performance of the robot
should be the determining factor in qualification. The RoboCup
Humanoid League TC discussed the issue and decided on a
compromise. The stated policy for qualification is that teams that
purchase a robot had to clearly highlight what advancements and
improvements they had made to the out of the box system for
qualification.

The introduction of the DARwIn-OP, however, also posed some problems
in the rule development. The RoboCup Humanoid League TC had a long
standing tradition of reducing the maximum allowed foot size of the
robots every two years. This reduction was in the rule appendix. The
implementation of that rule would have resulted that the DARwIn robots
would not be eligible for the 2013 competition. The trustees were
worried about possible repercussions from the Robotis company, who had
made a special offer for RoboCup teams and overruled the majority of
the RoboCup Humanoid League TC.

\section{Brazil Ole Ola (2014)}
\label{sec:four}

At the end of the 2013 RoboCup competition, the trustees issued a
challenge to all leagues as they felt that progress in the leagues had
been limited to incremental improvements rather than radical
breakthroughs.

After discussions with the team leaders, one major change was an
aggressive push towards larger robots. The 2014 competition introduced
radical changes in the sizes of the kid, teen, and adult leagues. For
example, the maximum height of the robots in the kid size was raised
by 50\% to 90cm. Furthermore, the height limits of the kid and teen
and the teen and adult sized leagues were chosen with some overlap on
the upper and lower limits. This allows teams more easily to move to
larger sub leagues, since they do not need to build a completely new
robot. For example, a team could build an 85cm tall robot and compete
in the kid size league in the first year and use the same robot in the
teen size league the next year.

The field dimensions, the size of the goals, and the size and weight
of the ball were adjusted to accommodate the larger robots.

Other changes try to enhance team play. The numbers of players for the
soccer matches was increased to 4 players per team.  Also, a robot is
not allowed to dribble the ball across the centre line, but the robot
must pass it to a team mate.

The complexities of the challenges also increased. The dribble and
kick competition for adult sized robots will introduce two obstacles
(representing opposing players) that must be avoided by the striker
robot.

\section{The Future of the Humanoid League (2014 - 2050)}
\label{sec:five}

The RoboCup Humanoid League continues its rapid advance towards
smarter and more capable robots. The goal is to move quickly towards
more realistic soccer matches. There are several extremely difficult
problems that need to be overcome with respect to the environment, the
players, and the matches.

One future direction of the RoboCup Humanoid League is to move to more
human like playing fields and environments. For example in 2015, the
RoboCup Humanoid League TC plans to reduce the colours in the
environment even further and to move towards requiring texture-based
segmentation. The plan is to remove the yellow goal posts and to
replace them with white goal posts with possibly black and white
textures. Furthermore, the orange ball will be replaced by a real
soccer ball, that is a ball that is mostly white or grey with some
texture to it.

Also, the playing surface will be changed from a carpeted floor to
astro turf and eventually a real grass playing field. This means that
active balancing and uneven terrain walking will become more important
for the robots. As the number and speed of the robots increases,
collisions between players are more likely to occur. Therefore, the
RoboCup Humanoid League TC will introduce push recovery challenges
where the robots the ability of the robots to compensate for pushes
from various directions.

In addition. corner flags similar to human soccer will be
introduced. These changes combined will make the playing fields in the
RoboCup Humanoid League scaled down versions of the human soccer
playing fields.

The progress in the teen sized league has shown that it is now
possible to have soccer matches for 80cm to 120cm tall humanoid
robots. But the ambitious goal of the RoboCup Humanoid League TC is to
introduce rules that move towards robots capable of playing against
human players. Therefore, the minimum height of the robots will be
increased in stages from the current 40cm to 140cm. As the capability
of the robots increases, the RoboCup Humanoid League will play with
larger and larger robots that become more and more similar to human
players in their kinematics.

Lastly, the rules of the game have to be adjusted to match exactly the
FIFA rules for human soccer. This requires that robots must be able to
act fully autonomously during all aspects of the game (including
kick-offs, substitutions, free kicks). Furthermore, the games will
include throw-ins and direct and indirect free kicks. It is
interesting to note that this rule progression towards more human like
soccer is not always linear. For example, free kicks were included in
the rules from 2004 to 2007, but slowed down the game
significantly. Therefore, all free kicks were replaced by 30 second
removal penalties similar to ice hockey rather than soccer. This made
games much more entertaining and made the RoboCup Humanoid League the
most exiting league in RoboCup. However, as teams improve the skills
and capabilities of their robot, free kicks can be re-introduced while
still resulting in exciting matches.

The number of players will be increased in the following years. We
plan on playing with 4 vs 4 players in 2015 and to eventually reach 11
vs 11 players in 2050. The RoboCup Humanoid League TC realizes that
few teams will be able to afford 11 players and has also started to
build the necessary technical infrastructure as well as amendments to
the rules to encourage joint or mixed teams.

The other issue with the robots in the RoboCup Humanoid League are
their robustness and energy efficiency. Currently the soccer games in
the RoboCup Humanoid League last only 20 minutes. In the coming year,
the duration of the games will be increased to match that of human
soccer matches -- 90 minutes per game.

\section{Risks and Issues}
\label{sec:seven}

This paper would be incomplete if it were not to include some words of
warning for the future development of the league. The initiatives
described in Sec.~\ref{sec:five} are far reaching and ambitious. As
such, it is clear that they entail a certain amount of risk.

The first issue is that moving to larger robots will greatly increase
the costs associated with RoboCup participation for all teams. Larger
robots require much more torque and thus much more expensive
motors. Furthermore, large robots cannot be brought in check-in
luggage and often must be shipped as cargo.

Combined with the ever increasing registration fees of RoboCup, this
may lead to teams deciding not to participate in RoboCup. For example,
several of the teams that participated in the RoboCup Humanoid League
for many years (e.g., Tamkung University, Damshui, Taiwan, NCKU,
Tainan, Taiwan, and Fumanoids, Berlin, Germany) will not participate
at RoboCup 2014 for financial reasons.

For example, the number of participants in the teen sized and adult
sized competition remained low with four to six teams each. The low
number of participants was in spite of the best efforts of the RoboCup
Humanoid League and the supportive attitude of the RoboCup Federation
in general to promote large humanoid robots.

Another problem is that the move to larger robots will make entry for
new teams much harder; the current road map does not provide a path
for new teams. One suggestion discussed by the RoboCup Humanoid League
TC was to keep the current kid size league, but to limit participation
in this league to two years. Another idea is to promote mentor teams
where more experienced teams form joint teams with less experienced
teams and provide them with technical support.

\section{Conclusions}
\label{sec:six}

The paper describes the history of the RoboCup Humanoid League from
its humble beginnings in 2002. It describes the historical evolution
of its competitions and rules to provide the reader with an insight
into the culture and traditions of the RoboCup Humanoid League. This
will make it easier to understand the current state as well as future
plans for the development of the RoboCup Humanoid League toward the
2050 goal.

The author would like to thank previous and existing members of the
RoboCup Humanoid League community for their input during many years of
rule discussions and development. In particular, I would like to thank
the current members of the RoboCup Humanoid League TC and organizing
committees (Sven Behnke, Reinhard Gerndt, Luis F. Lupian, Marcell
Missura, Daniel Seifert, Soroush Sadeghnejad, and RoboCup trustee
Oskar von Stryk)

\end{document}
